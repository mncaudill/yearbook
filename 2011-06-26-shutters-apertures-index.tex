\section{Shutters and Apertures}

\emph{Post published on 26 June 2011}

Through college, I used a little Canon PowerShot that more or less had
an on-and-off switch and handled all the messy details of focusing and
metering for me. For capturing events, it worked, but there was nothing
timeless or intriguing about the images I took.

A few years ago, I decided that I wanted to learn what made a photograph
``good'' and my first step was learning how cameras worked. I'm not
really comfortable with ``magic'' in most forms, so my wife snagged a
decades-old Minolta X--370 off Craigslist for a few bucks from a guy
with whiskey on his breath at a time of day when that sort of thing is
worrisome.

This kind of camera gives you full control over aperture size and
shutter speed, and while it does have a built-in meter, it forces you to
think about how light is hitting the film and adjust things to get the
aesthetic you want.

%\begin{wrapfigure}{o}{0.4\linewidth}
%\href{http://www.flickr.com/photos/spuklo/4543542297/}{\includegraphics[width=\linewidth]{farm5.static.flickr.com/4064/4543542297_7bc7ded751.jpg}}
%\end{wrapfigure}

The X--370, for me, was a great first ``real'' camera and I still shoot
with it occasionally. It shoots in aperture-priority mode, meaning that
all I control is how big the aperture opens and the camera adjusts the
shutter accordingly. Aperture is how you control your depth-of-field,
meaning if you want the nice effect of not having everything in equally
sharp focus, you can make the aperture larger.

My little Canon gave me little access to the aperture and thus my
pictures were all sharply in focus at all depths, so the person's face 6
feet away was just as sharp as the tree a few hundred feet behind him.
For some shots, this worked fine but sometimes you wanted to pull your
subject away from a busy background or draw attention to something in
frame by just using the depth of field.

With controlling just the aperture, I could choose which effect I wanted
and with this one change, I started liking my pictures more.

My current daily shooter is a Canonet QL17 G-III which is a
shutter-priority rangefinder. This means I tell the camera how fast to
run the shutter and it figures out how big to make the aperture, so a
faster shutter means less light hitting the film which means the
aperture needs to be open wider for a proper exposure.

Shutter speed is another major way to control the aesthetic of an image.
An oft-shot image using shutter styling is where someone shoots a river
but slows down the shutter speed enough to make the flowing water seem
more ``fuzzy'' than sharp. With fast-moving water, you capture more of
the movement of the water instead of how it looks at one instant in
time. Shooting a baseball player in the middle of a swing will be two
complete different pictures if you have a fast shutter speed versus a
slow shutter speed.

For me, I seldom take photographs where I try to play with getting
movement on film so I still use my shutter-priority camera like an
aperture-based one, adjusting the shutter speed until I get the aperture
size I want.

%\begin{wrapfigure}{o}{0.5\textwidth}
\href{http://www.flickr.com/photos/nolancaudill/5861502081/}{\includegraphics[width=\linewidth]{farm3.static.flickr.com/2736/5861502081_2ab2ec15ed.jpg}}
%\end{wrapfigure}

The above photograph was taken with the aperture somewhere between f/8
and f/11, if I remember correctly. It was a really sunny day and I
wanted to get the people exposed as well as I could while still getting
everything in focus, but with the direct sunlight, I couldn't point and
click without completely underexposing the people, rendering them as
dark silhouettes. So, I filled my frame with people in the foreground,
and adjusted the shutter speed until I got a small enough aperture for
the depth-of-field I wanted, and I then set this aperture manually,
overriding the automatic mode. I then framed the whole scene including
the sun, fairly confident that most of the people wouldn't be
underexposed.

%\begin{wrapfigure}{o}{0.5\linewidth}
\href{http://www.flickr.com/photos/nolancaudill/5862044848/}{\includegraphics[width=\linewidth]{farm4.static.flickr.com/3071/5862044848_444914da61.jpg}}
%\end{wrapfigure}

This picture of Meghan getting ready in the morning uses a shallow
depth-of-field to draw the eye to her reflection as the most important
thing in the image. I sped the shutter speed up as much as possible to
make the aperture open as much as I could, knowing that if I focused on
her reflection, it would be sharply focused while her actual body would
not be as sharp.

\begin{wrapfigure}{o}{0.5\linewidth}
\href{http://www.flickr.com/photos/nolancaudill/5861982922/}{\includegraphics[width=\linewidth]{farm6.static.flickr.com/5116/5861982922_b2fe8f2527.jpg}}
\end{wrapfigure}

I took this picture of Trevor and Bert at a good Japanese restaurant
near Union Square. In this image, I noticed that the menus and posters
on the back wall were fairly noisy and if I shot anything smaller than
f/4 or f/5.6, these posters would be sharply focused and would clutter
the scene. Speeding the shutter up as much as I could in this underlit
restaurant opened the aperture up wide, which put the guys in sharp
focus while fading out what was in the background.


To play with this yourself if you have a digital camera, put your camera
in something like Program mode which will choose a ``best''
shutter-and-aperture combination but lets you change each independently
while keeping it exposed properly. This way you can see exactly the
effect that shutter speed and aperture size have on the final image.
